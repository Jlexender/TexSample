\section{2023-10-20}

\begin{definition}[]
	$e = \lim_{n \to \infty} \left( 1 + \frac{1}{n} \right) ^n$
\end{definition}

\begin{lemma}
	$e = 1 + \frac{1}{1!} + \frac{1}{2!} + \dots + \frac{1}{n!} + \frac{\theta}{n \cdot n!}$
\end{lemma}
\begin{proof}[Доказательство]
	$\lim_{n \to \infty} e_n = e$. With fixed n: \[
		e - e_n = \lim_{k \to \infty} e_k - e_n = \lim_{k \to \infty} \frac{1}{(n+1)!} + \dots + \frac{1}{k!}
	\]

	$k = n + p, p \in \N$

	$
		\frac{1}{(n+1)!} + \dots + \frac{1}{(n+p)!} = \frac{1}{(n+1)!} \left( 1 + \frac{1}{n+2} + \\ \frac{1}{(n+2)(n+3)} + \dots + \frac{1}{(n+2)(n+3) \dots (n+p)} \right) < \frac{1}{n \cdot n!}
	$
\end{proof}

\begin{corollary}[]
	$e \in I$
\end{corollary}

\begin{proof}[Доказательство]
	Let $e = \frac{m}{n} \Rightarrow e = 1 + \frac{1}{1!} + \frac{1}{2!} + \dots + \frac{1}{n!} + \frac{\theta}{n \cdot n!} \Rightarrow m(n-1)! = n! + n! + \frac{n!}{2} + \dots + \frac{n!}{n!} + \frac{\theta}{n} !?$
\end{proof}

\subsection{Subsequences}

Let $a_n$ be a subsequence.

\begin{definition}[]
	Let $n_k: n_1 < n_2 < \dots < n_{k+1} < \dots $

	$a_{nk}$ is a subsequence of $a_n$
\end{definition}

\begin{definition}[]
	If $\exists \lim_{k \to \infty} a_{nk} = A \in \overline{\R}$, then it's called partial limit.
\end{definition}

\begin{example}[]
	\begin{enumerate}
		\item $a_n = (-1)^{n}: a_{2k} = 1, a_{2k-1} = -1$
		\item $a_{n} = n^{(-1)^{n}}: a_{2k} = 2k, a_{2k-1} = \frac{1}{2k-1}$ 
		\item $a_n = 1,1,2,1,2,3,1,2,3,4, \dots $. $\N \cup +\infty$
		\item $a_n: 1, \frac{1}{2}, \frac{1}{3}, \frac{2}{3}, \frac{2}{4}, \dots $. $\left[ 0,1 \right]$ 
	\end{enumerate}
\end{example}

\begin{lemma}
	If $\lim_{n \to \infty} a_n = A \in \overline{\R}$, then $\forall a_{nk}: \lim_{k \to \infty} a_{nk} = A$
\end{lemma}

\begin{proof}[Доказательство]
	$\forall \epsilon > 0 \exists n_0: \forall n \geq n_0: a_n \in U_{\epsilon}(A)$. $n_k \uparrow$ and not bounded above, then  $\exists k_0: n_{k_0} \geq n_0 \Rightarrow \forall k \geq k_0 a_{nk} \in U_\epsilon(A) \Rightarrow \lim_{k \to \infty} a_{nk} = A$.
\end{proof}

\begin{theorem}[Больцано-Вейерштрасса]
	\begin{itemize}
		\item We can choose a subsequence that has a limit in \textbf{any} limited sequence.
	\end{itemize}
\end{theorem}

\begin{proof}[Доказательство]
	\begin{enumerate}
		\item Множество значений последовательности конечно. $ \Rightarrow $ хотя бы 1 значение встречается бесконечное число раз $ \Rightarrow \exists a_{nk} \equiv \text{ const } $.
		\item Infinite set of values.

			Bounded $ \Rightarrow \exists \text{ a limit point of X } \Rightarrow \forall  \overset{o}U_\epsilon (x) \text{ there is infinite amount of elements }  $ 

			Let $\epsilon = 1 \Rightarrow a_{n_1} =  \overset{\circ}U _1 (x) $

			Let $\epsilon = \frac{1}{2} \Rightarrow \exists n_2 > n_1: a_{n_2} \in  \overset{\circ}U_{\frac{1}{2}}(x) $

			$\vdots$

			 $\epsilon = \frac{1}{k} \Rightarrow \exists n_k > n_{k+1}: a_{n_k} \in  \overset{\circ}U_{\frac{1}{k}} (x) $

			 $ \Rightarrow  \exists k: \left| a_{n_k} - x \right| < \frac{1}{k} < \epsilon \Rightarrow a_{n_k} \mapsto x $
	\end{enumerate}
\end{proof}

\begin{corollary}[]
	$\forall a_n : $ a set of partial limits is not empty.
\end{corollary}

\begin{definition}[]
	Upper limit is the biggest partial limit. Same for lowest
\end{definition}

$\limsup_{n \to \infty} a_{n}, \liminf_{n \to \infty} a_n$



\begin{theorem}[closedness of set of partial limits]
	Let $a_n$ be a limited set, $E$ is a set of partial limits. 
\end{theorem}

\begin{proof}[Доказательство]
	Let $x$ be a limit point of $E$. Then $ \forall \epsilon > 0 \exists a \in E: a \in \overset{o}U_\epsilon  (x), \text{ i.e. } \left| x - a \right| < \frac{\epsilon}{2} \Rightarrow \exists a_{nk} \to a$. Consider $\left| a_{nk} - x \right|  \leq \left| a_{nk} - a \right| + \left| x - a \right| < \epsilon $. Then $x \in E \Rightarrow E \text{ is closed } $ 
\end{proof}
