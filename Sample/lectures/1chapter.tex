\section{2023-09-04}
\subsection{Intro}
Lecturer: Trifanova Ekaterina Stanislavovna

Telegram: @estrifanova

The whole semester is splitted by 3 parts, each of them contains: 
\begin{enumerate}
	\item control test 
	\item theory test (on GeoLin) 
	\item hometask
	\item colloquium 
\end{enumerate}

\textbf{Summary: 100 p.}

The course is linked to A. Boytsev's course.

\subsection{Logical symbolic}

\begin{definition}[]
	A statement is a sentence that is either \textbf{true} or \textbf{false}.
\end{definition}
 
\begin{itemize}
	\item $\forall$ -- for all 
	\item $\exists$ -- exists 
	\item ! -- single
	\item $\sqsupset$ -- let
\end{itemize}

\begin{example}[]
	$\forall a \in \N \ \exists ! \ b \in \N: a + b = 0 $
\end{example}

$A \Rightarrow B, A \Leftarrow B, A \Leftrightarrow B$
\[
	A = "a \ \vdots \ 6" \\
	B = "a \ \vdots \ 3" \\
	A \implies B
\]

$\wedge \text{ -- conjunction}, \vee \text{ -- dis junction}$. 
\[
	A \iff B \vee C
\]

\begin{lemma}
	A is true $\iff$ $\neg$ A is false
\end{lemma}

\pagebreak
\subsection{Sets}


\begin{definition}[]
	A set is a group of an objects, with defined rule which can define, is object in a set or not 
\end{definition}

Парадокс Рассела (множество множеств, которое не содержит себя в качестве элемента)

$a \in A$

$b \notin A \iff \neg(b \in A)$

\begin{definition}[]
	$A \subset B \iff \forall a \in A \implies a \in B$
\end{definition}
\begin{definition}[]
	$A = B \iff A \subset B \wedge B \subset A$
\end{definition}

\subsection{Operations between sets}

\begin{enumerate}
	\item Union -- $A \cup B = \{x: x \in A \vee x \in B\}$
	\item Intersection -- $A \cap B = \{x: x \in A \wedge x \in B\}$
\end{enumerate}

Let $A$ be a set of indexes. $\alpha \in A. \ \alpha \mapsto G_\alpha$. We want do define a union of $n$ sets.

\begin{definition}[]
\[
	\bigcup_{\alpha \in A} G_\alpha = \{x : \exists \alpha: x \in G_\alpha\} = G_{A_1} \cup G_{A_2} \cup G_{A_3} \cup \dots \cup  G_{A_n} 
\]
\end{definition}

\begin{definition}[]
\[
	\bigcap_{\alpha \in A} G_\alpha = \{x : \forall \alpha: x \in G_\alpha\} = G_{A_1} \cap G_{A_2} \cap G_{A_3} \cap \dots \cap  G_{A_n} 
\]
\end{definition}

\begin{definition}[]
\[
	A \ \backslash \ B = \left\{ x: x \in A \wedge x \notin B \right\} 
\]
\end{definition}

We define $U$ is an universal set: $\forall x: x \in U$. 

$U \ \backslash \ A = A^{c} $ -- complement ion.

$A \times B$ -- desert multiplication of sets. $A \times B = \left\{ (x, y): x \in A, y \in B \right\} $

\begin{lemma}[Свойства операций]
	$\forall A, B, C$:
	 \begin{enumerate}
		\item $A \cup B = B \cup A$ (коммутативность)
		\item $A \cap B = B \cap A$ (коммутативность)
		\item $A \cup (B \cup C) = (A \cup B) \cup C$ (ассоциативность)
		\item $A \cap (B \cap C) = (A \cap B) \cap C$ (ассоциативность)
		\item $A \cup A = A \cup \varnothing = A$
		\item $A \cap \emptyset = \emptyset$
		\item $A \cap A = A$
		\item $A \cup A^{c} = U$
		\item $A \cap A^{c} = \emptyset$
		\item $(A^{c})^{c} = A$
	\end{enumerate}
\end{lemma}

\begin{exercise}
	Доказать верхние 10 свойств. Доказательство достаточно тривиально.
\end{exercise}
\begin{note}
	Доказательство следует из определения.
\end{note}

\begin{theorem}[великая теорема Ферма]
	$\forall x,y,z \in \Z: x,y,z > 2: x^{n} + y^{n} = z^{n}$ не имеет решений.	
\end{theorem}
