\section{2023-11-03}

Consider $1 + \frac{1}{2} + \frac{1}{3} + \dots + \frac{1}{n}$. $y_n = \ln (n+1)$

\begin{lemma}
	$\exists \lim_{n \to \infty} \left( 1 + \frac{1}{2} + \dots + \frac{1}{n} - \ln n \right) = \gamma \approx 0,577 $
\end{lemma}

\begin{proof}[Доказательство]
	Let $\gamma_n = 1 + \dots + \frac{1}{n} - \ln n$. Proving by Weierstrass. Consider $\gamma_{n+1} - \gamma_n = \frac{1}{n+1} - \ln \left( n + 1 \right)  + \ln n = \frac{1}{n+1} - \ln \left( 1 + \frac{1}{n} \right) < 0 \Rightarrow \gamma_n \downarrow$.

	Consider $\overline{\gamma_n} = 1 + \dots + \frac{1}{n} - \ln \left( n+1 \right). \gamma_n - \gamma_{n+1} = \frac{1}{n}- \ln \left( 1 + \frac{1}{n} \right) > 0 \Rightarrow \overline{\gamma_n} \uparrow $. But $\overline{\gamma_n} = \gamma_{n+1} - \frac{1}{n+1} < \gamma_{n+1} \Rightarrow \text{ by Weierstrass } \exists \lim_{n \to \infty} \gamma_n, \overline{\gamma_n} \text{ and  } \lim_{n \to \infty} \gamma_n = \lim_{n \to \infty} \overline{\gamma_n}$
\end{proof}

\begin{note}[]
	\[
		1 + \frac{1}{2} + \dots + \frac{1}{n} = \ln n + \gamma + \frac{\theta}{n}, \theta \in \left[ 0, 1 \right]
	\]
\end{note}

\begin{proof}[Доказательство]
	Consider $0 \leq \gamma_n - \gamma \leq \gamma_n - \overline{\gamma_{n-1}} = \frac{1}{n}$. Then $0 \leq R \leq \frac{1}{n} \Rightarrow \exists \theta \in \left[ 0, 1 \right]: R = \frac{\theta}{n}$
\end{proof}

\subsection{CH. 2. FUNCTION LIMIT}

\subsection{Definitions}

Let $f: \R \supset E \to \R, x_0 \text{ is a limit point }, a_n: \N \to \R $.

\begin{definition}[Cauchy]
	$\lim_{x \to x_0} f(x) = A \in \R, x_0 \in \R \Leftrightarrow \forall \epsilon > 0 \exists \delta(\epsilon) > 0: \forall x \in E, 0 < \left| x - x_0 \right| < \delta \Rightarrow \left| f(x) - A \right| < \epsilon$
\end{definition}

\begin{definition}
	$\forall \epsilon > 0 \exists \delta > 0: \forall x \in E \cap \overset{\circ} U_\delta (x_0): f(x) \in U_\epsilon (A) $. Works for $x_0, A \in \overline{\R}$
\end{definition}

\begin{example}[]
	$\lim_{x \to -\infty} f(x) = +\infty \Leftrightarrow \forall \epsilon > 0 \exists \delta > 0: \forall x \in E, x < -\frac{1}{\delta} \Rightarrow f(x) > \frac{1}{\epsilon}$
\end{example}

\begin{note}[]
	$\lim_{x \to x_0} f(x) = \infty \Leftrightarrow \lim_{x \to x_0} \left| f(x) \right| = +\infty$
\end{note}

\begin{definition}[topological]
	$\displaystyle \lim_{x \to x_0} f(x) = A, x_0, A \in \R \Leftrightarrow \forall V(A) \exists \overset{\circ} U (x_0): Im \left( \overset{\circ} U (x_0) \cap E \right) \subset V(A)$
\end{definition}

\begin{exercise}
	Докажите равносильность.
\end{exercise}

\begin{definition}[ГЕЙне]
	$\lim_{x \to x_0} f(x) = A; x_0, A \in \overline{\R} \Leftrightarrow \forall x_n: x_n \in E, x_n \neq x_0, x_n \to x_0 \Rightarrow \lim_{n \to \infty} f(x_n) = A$
\end{definition}

\begin{theorem}[iff]
	\begin{enumerate}
		\item K to G is obv.
			\begin{proof}[Доказательство]
				Let $x_n \in E, x_n \neq x_0, x_n \to x_0$. For $\delta > 0 \exists n_0: \forall n \geq n_0: x_n \in U_\delta(x_0) \overset{x_n \neq x_0} \Rightarrow x_n \in \overset{\circ} U_\delta(x_0) \cap E \overset{Cauchy \ def.} \Rightarrow f(x_n) \in U_\epsilon(A) \Rightarrow \lim_{n \to \infty} f(x_n) = A$
			\end{proof}
		\item G to K.
			\begin{proof}[Доказательство]
				Let $\forall x_n: x_n \in E x_n \neq x_0, x_n \to x_0 \Rightarrow f(x_n) \to A$. let Cauchy be false then $\exists \epsilon > 0: \forall \delta > 0 \exists x \in E \cap \overset{\circ} U_\delta (x_0): f(x) \notin U_\epsilon(A)$. Let $\delta = \frac{1}{k}, k \in \N \Rightarrow \exists x_k \in E \cap \overset{\circ} U_\delta(y_k): f(x_k) \notin U_\epsilon(A)$. Consider $x_k: x_k \in E, x_k \neq x_0 0 < \left| x_k - x_0 \right| < \frac{1}{k} \Rightarrow x_k \to x_0 \Rightarrow f(x_k) \to A$
			\end{proof}
	\end{enumerate}
\end{theorem}
