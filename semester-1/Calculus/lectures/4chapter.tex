\section{2023-09-15}

\subsection{Expanding $\R$}

\begin{definition}[$\overline{\R}$]
	$\overline{\R} = \R \cup \left\{ +\infty, -\infty \right \} $.
\end{definition}

\begin{property}[]
	$\forall x \in \R$:
	\begin{itemize}
		\item $x + (+\infty) = +\infty := x + \infty$
		\item $x + (-\infty) = -\infty := x - \infty$
		\item $x \cdot (\pm \infty) = \begin{cases}
			\pm \infty, &\text{ if } x > 0\\
			\mp \infty, &\text{ if } x < 0\\
			undefined , &\text{ if } x  = 0\\
			\end{cases}$ 
		\item $\frac{x}{\pm \infty} = 0$  
		\item $\frac{\pm \infty}{x} = \begin{cases}
			\pm \infty, &\text{ if } x > 0\\
			\mp \infty, &\text{ if } x < 0\\
		\end{cases}$ 
	\end{itemize}
	
	\noindent
	$(+\infty) + (+\infty) = +\infty \\ (-\infty) + (-\infty) = -\infty \\ (+\infty) \cdot (+\infty) = (-\infty) \cdot (-\infty) = +\infty \\ (+\infty) \cdot (-\infty) = (-\infty) \cdot (+\infty) = -\infty$

	$\forall x: -\infty < x < +\infty$

	Actions undefined in $\R$:
	\begin{itemize}
		\item $0 \cdot (\pm \infty)$
		\item $(+\infty) + (-\infty)$
		\item $1^{\infty}$ 
		\item $\frac{\pm \infty}{\pm \infty}$ 
		\item $\frac{0}{0}$ 
		\item  $0^{0}$
	\end{itemize}
\end{property}

\subsection{Defining $\N$}

\begin{definition}[Inductive set]
	A set $X \subset \R$ is \textit{inductive}, if $\forall x \in X: x + 1 \in X$
\end{definition}

\begin{lemma}
	Let $X_1, X_2, \dots , X_n$ be inductive sets. Then, $X_1 \cap X_2 \cap \dots X_n$ is also inductive.
\end{lemma}

\begin{proof}[Доказательство]
	Trivially proof the $x \mapsto x+1$
\end{proof}
\begin{definition}[]
	$\N$ is an intersection of every inductive sets: $\forall i: 1 \in A_i$
\end{definition}
\begin{note}[]
	$\N$ is minimal inductive set, that contains 1.
\end{note}

\begin{theorem}[Math. induction principle]
	Let $X \subset \N, 1 \in X, X$ is inductive. Then, $\N = X$
\end{theorem}

\begin{exercise}
	Proof that $\forall n > -1, n \in \N, x \in \R: (1 + x)^{n} \geq 1 + nx$
\end{exercise}

\subsection{Properties of $n \in \N$}

\begin{lemma}
	$\forall a,b \in \N: a + b \in \N, ab \in \N$
\end{lemma}

\begin{note}[]
	Proof using math. induction.
\end{note}

\begin{definition}[$\Z$]
	$\Z := \N  \cup \left\{ 0 \right \} \cup \{x : -x \in \N\}$
\end{definition}

\begin{definition}[$\Q$]
	$\Q := \left\{ \frac{m}{n} := m \cdot n^{-1}, m \in \Z, n \in \N \right \} $	
\end{definition}

\begin{theorem}[Existence of irrational number]
	A set $\R \ \backslash \ \Q = \mathbb{I}$ is not empty.
\end{theorem}

Let's proof that $\sqrt{2}$ is irrational.
\begin{proof}[Доказательство]
	Plan:
	\begin{enumerate}
		\item Prove that $\exists c \in \R: c^2=2$.
		\item  Prove that $c$ is irrational.
	\end{enumerate}	
	\begin{enumerate}
		\item [2. ] Let $c = \frac{m}{n}, m \in \Z, n \in \N$. Then $c^2 \cdot n^2 = m^2 \Rightarrow 2n^2 = m^2 !?$
		\item [1. ] Using axiom of continuity. Let $X = \left\{ x \in \R_{x > 0}: x^{2} < 2 \right \} , Y = \left\{ y \in \R_{y > 0}: y^2>2 \right \} $. Then $x \leq y \Rightarrow \exists c \in \R: x \leq c \leq y \forall x \in X, y \in Y$ 
			
	Proving that $c \notin X$. Let  $c \in X$, i.e. $c^2 < 2$. Consider $c + \frac{2 - c^2}{3c} = c + \frac{\Delta}{3c} = \xi \\ (c + \frac{\Delta}{3c})^2 = c^2 + \frac{2}{3}\Delta + \frac{\Delta \cdot \Delta}{9c^2} \leq c^2 + (\frac{2}{3} + \frac{1}{3})\Delta = 2 \Rightarrow \xi \in X, \text{ but } \xi > c !? \Rightarrow c \notin X$. Similarly, we proof for $Y$.
	\end{enumerate}
	$ \Rightarrow \exists c \in \R: c^2 = 2 \Rightarrow |\mathbb{I}| \neq 0$
\end{proof}
