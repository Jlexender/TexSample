\section{2023-09-11}

$ \triangleleft $ talked about mappings (and will be in the 1st semester)

\subsection{Defining $\R$}

Мы выбираем \textit{аксиоматический} подход. 

\begin{definition}[$\R$]
	We call a set an $\R$ if:
	\begin{itemize}
		\item Addition
			\begin{enumerate}
				\item[def] $"+" : \R \times \R \mapsto \R$ is satisfied:
				\item (commutativity): $a + b = b + a$
				\item (associativity): $a + (b + c) = (a + b) + c$ 
				\item $\exists 0: \forall a + 0 = a$. We call 0 a \textbf{neutral} element. 
				\item $\forall a \in \R : \exists (-a): a + (-a) = 0$ 
			\end{enumerate}
		\item Multiplication 
			\begin{enumerate}
				\item [def] $"\cdot" : \R \times \R \mapsto \R$ is satisfied:
				\item (commutativity): $a \cdot b = b \cdot a$
				\item (associativity): $a \cdot (b \cdot c) = (a \cdot b)\cdot c$
				\item $\exists 1 \neq 0: \forall a \in A: a \cdot 1 = a$
				\item $\forall a \in A: \exists a^{-1} \in A: a \cdot a^{-1} = 1$ 
			\end{enumerate}
		\item (distributivity): $\forall a,b,c \in \R: a \cdot (b+c) = a \cdot b + a \cdot c \ \& \ (a+b) \cdot c = a \cdot c + b \cdot c$ 
		\item (axioms of order) $\forall a,b \in \R$ mapping of order $ \leq $ set if:
			\begin{enumerate}
				\item $x \leq x$
				\item $(x \leq y \wedge y \leq x) \Rightarrow x = y$  
				\item (transitivity) $x \leq y \wedge y \leq z \Rightarrow x \leq z$ 
				\item $\forall x,y \in \R: x \leq y \vee y \leq x$ 
			\end{enumerate}
		\item (Connection between $ \leq , + $) $\forall x,y,z \in \R: x \leq y \Rightarrow x + z \leq y + z$ (this is not implied by previous conditions)
		\item (Connection betwen $\cdot$ and $ \leq $): $0 \leq x \wedge 0 \leq y \Rightarrow 0 \leq x \cdot y$  
		\item (Axiom of continuity (completeness)): Let $X, Y \subset \R: \forall x \in X: \forall y \in Y: x \leq y$. Then $\exists c \in \R: x \leq c \leq y$
		\begin{example}[This axiom doesn't work on $\Q$]
			Let $X = \left\{ x \in \Q: x \cdot x \leq 2  \right \}, Y = \left\{ y \in \Q: y \cdot y \geq 2 \right \}  $. Then, $\exists ! a \notin \Q \ \ (a = \sqrt{2}): $ satisfies this axiom.
		\end{example}
	\end{itemize}
\end{definition}

\begin{note}[]
	Definition of $\R$ just contains the conditions that satisfy the \textbf{field}.
\end{note}

\subsection{Corrolaries}

\begin{corollary}[Corrolaries on Axioms 1--3]
	$\\$
	\begin{enumerate}
		\item $\exists ! 0, \exists ! 1$.
			\begin{proof}[Доказательство для 0]
				Let there be $0_1, 0_2$. Then:  \[
					0_1 = 0_1 + 0_2 = 0_2
				\]	
			\end{proof}
		\item $\exists ! (-x) \forall x$
		\item $\forall x \neq 0 \exists ! x^{-1}$ 
			\begin{proof}[Доказательство]
				Let there be $-x_1$ and $-x_2$. Then:  \[
					(-x_1) = (-x_1) + (x + (-x_2)) = (x + (-x_1)) + (-x_2) = (-x_2)
				\]	
			\end{proof}
		\item $\forall a,b \in \R$ an equality $x + a = b$ is set. Then there is only one solution $x = b + (-a)$. 
		\item $x \cdot a = b (a,b \in \R)$. Then, $\exists ! x = b \cdot a^{-1}$ 
		\item $\forall x: x \cdot 0 = 0$ 
			\begin{proof}[Доказательство]
				$x \cdot 0 = x \cdot (0 + 0) = 0 \cdot x + 0 \cdot x = 0 \Rightarrow 0 = x \cdot 0$
			\end{proof}
		\item $x \cdot y = 0 \Leftrightarrow x = 0 \vee y = 0$ 
			\begin{proof}[Доказательство]
				$ \Leftarrow$ is proven.

				$ \Rightarrow : x \neq 0 \Rightarrow \exists x^{-1}: x \cdot y \cdot x^{-1} = 0 \Rightarrow y = 0$. Proof for $y$ is similar.
			\end{proof}
		\item $-x = -1 \cdot x $ 
			\begin{proof}[Доказательство]
				$-1 \cdot x + x = -1 \cdot x + 1 \cdot x = x(-1 + 1) = x \cdot 0 = 0 $
			\end{proof}
		\item $-1 \cdot (-x) = x$. Proof is trivial based on previous) 
		\item $(-x) \cdot (-x) = x \cdot x$. Proof is also trivial. 
	\end{enumerate}
\end{corollary}

\begin{definition}[]
	$$x \leq y \Leftrightarrow \geq x$$

	$$x < y \Leftrightarrow x \leq y \wedge x \neq y$$

	$$x > y \Leftrightarrow y \geq x \wedge y \neq x$$
\end{definition}

\begin{corollary}[Corrolaries on axioms 4 -- 6]
	$\\$
	\begin{enumerate}
		\item $\forall x,y \in \R$: the only one statement is true
			\begin{itemize}
				\item $x < y$
				\item $x = y$
				\item $x > y$ 
			\end{itemize}
		\item $x < y \wedge y \leq z \Rightarrow x < z$ 
		\item $\dots$ 
		\item $x > 0 \Leftrightarrow -x < 0$. The proof is obvious.
		\item $x < 0 \wedge y < 0 \Rightarrow xy > 0$ 
		\item Can add to strict inequality.
		\item $x \leq y \wedge z \leq w \Rightarrow x + z \leq y + w$ 
		\item $0 < x \wedge 0 < y \Rightarrow 0 < xy$ 
		\item $0 < x \wedge y < z \Rightarrow xz < yz$ 
		\item $1 > 0$
			\begin{proof}[Доказательство]
				Let $1 \leq 0 \Rightarrow 1 < 0 \Rightarrow 1 \cdot 1 > 0 !?$. Then, $1 > 0$.
			\end{proof}
	\end{enumerate}
\end{corollary}
