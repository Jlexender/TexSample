\section{2023-09-15}

\subsection{root of complex number}

\begin{itemize}
	\item let $a \in \Cm, n \in \N$. Solve $z^{n} = a$ 
	\item $a = (r, \phi), z = (\rho, \psi)$.
	\item By Moaur formula, $\rho = \sqrt[n]{r}$ 
	\item $n\psi =\phi + 2\pi k, k \in \Z$. Dividing by $n$, we get:
		\[
			\psi = \frac{\phi}{n} + \frac{2\pi k}{n}
		\]
	\item For $k \in \left\{ 0,1, \dots , n-1 \right \} $, we get $n$ different arguments.
	\item Every $k$ can be factorized as $k = qn + r$. Then  $\frac{2\pi k}{n} = \frac{2\pi r}{n} + 2\pi q$ 
\end{itemize}

\subsection{root of n for 1}

\begin{itemize}
	\item Consider $z^{n} = 1$
	\item From last section we get: $\psi_k = \frac{2\pi k}{n}$, where $k \in \left\{ 0,1, \dots , n-1 \right \} $ 
	\item Using Moaur formula $\epsilon_k = \epsilon_1^k$. Then, all roots of 1 is powers  $\epsilon_1$ 
\end{itemize}

\begin{note}[]
	$e ^{i\phi} = (\cos \alpha, \sin \alpha)$
\end{note}

\href{http://logic.pdmi.ras.ru/~dvk/ITMO/Algebra}{Materials} 

\subsection{Integers}

\subsection{divisibility}

\begin{definition}[]
	Let $a,b \in \Z, b \neq 0$. Then $a \vdots b$ or $b | a$, if  $a = bc$, where  $c \in \Z$
\end{definition}

\begin{property}[]
	If $a \vdots b, b \vdots c \Rightarrow a \vdots c$
\end{property}

\begin{proof}[Доказательство]
	Then $a = kb, b = nc (k,n \in \Z) \Rightarrow a = knc$.
\end{proof}

\begin{property}[]
	Let $a,b \vdots d, x,y \in \Z$. Then $ax + by \vdots d$
\end{property}

\begin{proof}[Доказательство]
	Then $a = kd, b = nd \Rightarrow ax + by = (kx + ny)d$
\end{proof}

\begin{property}[]
	Let $a,b \in \N, a \vdots d \Rightarrow a \geq d$.
\end{property}

\begin{theorem}[]
	Let $a \in \Z, b \in \N \Rightarrow \exists ! q,r \in \Z: 0 \leq r < b \ \& \ a = bq + r$
\end{theorem}

\begin{proof}[Доказательство]
	\begin{itemize}
		\item $\exists $. Let $q$ be an integer that $bq \leq a < b(q+1) $ and $r = a - bq$. Then  $0 \leq r < b$ 
		\item !. Let $a = bq_1 + r_1 = bq_2 + r_2$, where $0 \leq r_1, r_2 < b$ 
		\item Не умаляя общности $r_1 > r_2 \Rightarrow 0 < r_1 - r_2 < b$
		\item From other side, $r_1 - r_2 = b(q_2 - q_1) \geq b!?$
	\end{itemize}
\end{proof}

\subsection{GCD}

\begin{definition}[]
	Let $a_1, \dots , a_{n} \in \Z$. Denote $OD(a_1, \dots , a_{n})$ as a set of every divisors of these numbers. GCD is denoted as $(a_1, \dots , a_{n})$
\end{definition}

\begin{property}[]
	If $b \in \N, a \vdots b \Rightarrow OD(a,b)$ is all divisors of $b \text{ and } (a,b) = b$
\end{property}

\begin{proof}[Доказательство]
	\begin{itemize}
		\item If $d$ is common divisor of $a,b$, then $d | b$.
		\item If $d | b$, then $a \vdots d$ using property 1 of divisibility. That means that $(a,b) = d$. 
	\end{itemize}
\end{proof}

\begin{property}[]
	let $a,b,c,k \in \Z, c = a + kb$. Then $OD(a,b) = OD(c,b) \Rightarrow (a,b) = (c,b)$

\end{property}

\begin{proof}[Доказательство]
	\begin{itemize}
		\item Let $d \in OD(a,b)$. Then $c \vdots d \Rightarrow d \in OD(c,b)$
		\item If $d \in OD(c,b)$, then $a = c - kb \vdots d \Rightarrow d \in OD(a,b)$ 
	\end{itemize}
\end{proof}

\subsection{Euclid algorithm}

\begin{itemize}
	\item Let $a,b \in \N, a > b$
		\begin{enumerate}
			\item [1] $a = bq_1 + r_1$
			\item [2] $b = r_1q_2 + r_2$
			\item [3] $r_1 = r_2q_3 + r_3$
			\item $ \dots $
			\item [n] $r_{n-2}$
		\end{enumerate}
	\item $b > r_1 > r_2 > \dots $ and algorithm will stop. 
\end{itemize}

\begin{theorem}[]
	$(a,b) = r_n \ \& \ OD(a,b)$ are all $r_n$ divisors
\end{theorem}

\begin{proof}[Доказательство]
	Using Euclid algorithm 
\end{proof}

\begin{theorem}[]
	Let $a,b,m \in \N$. Then
	\begin{enumerate}
		\item $(am, bm) = m(a,b)$
		\item if $d \in OD(a,b)$, then $(\frac{a}{d}, \frac{b}{d}) = \frac{(a,b)}{d}$ 
	\end{enumerate}
\end{theorem}

\begin{note}[]
	Using Euclid algorithm, basing on 1st line of algorithm.
\end{note}

\begin{exercise}
	Proof the theorem above.
\end{exercise}

\subsection{Linear GCD representation}

\begin{theorem}[]
	Let $a,b \in \Z$. Then $\exists x,y \in \Z: (a,b) = ax + by$ 
	\begin{itemize}
		\item It's called linear representation of GCD.
	\end{itemize}
\end{theorem}

\begin{proof}[Доказательство]
	\begin{itemize}
		\item $GD(y) = GD(-y) \Rightarrow (a,b) = (a, -b)$. Then we assume that $a,b \in \N$.
		\item HYO  $a \geq b$. Using Euclid algorithm, let $r_0 = b, r_{-1} = a$
		\item Prove that $(a,b) = x_k r_k + y_k r_{k-1} \forall k = \left\{ n, \dots , 0 \right \} $ (where $(a,b) = r_{n}$) by induction 
		\item $k = n$ is obvious.
		\item $k \mapsto k-1$. We know that $r_k = r_{k-2} + r_{k-1} q_k$:
	\end{itemize}
\end{proof}

\subsection{GCD of n numbers}

\begin{theorem}[]
	Let $n \geq 2, a_1, \dots , a_n \in \Z$. Puts $m_2 = (a_1, a_2), m_3 = (m_2, a_3), \dots , m_n = (m_{n-1}, a_n)$. Then $m_n = (a_1, a_2, \dots , a_n)$, and $OD(a_1, \dots , a_n)$ are all $m_n$ divisors.
\end{theorem}

\begin{proof}[Доказательство]
	Using induction (trivial)
\end{proof}

\begin{corollary}[]
	for $a_1, \dots , a_n \in \Z \exists \text{ linear representation of GCD }: x_1, \dots , x_n \in \Z: (a_1, \dots , a_n)= x_1a_1+ \dots + x_n a_n $
\end{corollary}

\begin{proof}[Доказательство]
	Trivial proof using induction.
\end{proof}

\subsection{relatively prime numbers}

\begin{definition}[]
	\begin{itemize}
		\item  $a_1, \dots , a_n \in \Z: (a_1, \dots , a_n) = 1 \Rightarrow $ these are relatively prime 
		\item Попарно взаимно простые 
	\end{itemize}
\end{definition}

\begin{property}[]
	If  $a,b,c \in \Z \ \& \ (a,b) = 1 \Rightarrow (ac, b) = (c, b)$
\end{property}

\begin{proof}[Доказательство]
	\begin{itemize}
		\item Let $d = (c,b) \ \& \ f = (ac, b)$
		\item $c \vdots d \Rightarrow ac \vdots d \Rightarrow d \in OD(ac, b) \Rightarrow f \vdots d$
		\item $b \vdots f \Rightarrow bc \vdots f \Rightarrow f \in OD(ac, bc)$
		\item $ \Rightarrow ^{\text{ th 2, 3 } } c = c(a,b) = (ac, bc) \vdots f$ 
		\item $ \Rightarrow, f \in OD(c,b) \Rightarrow ^{\text{ th. 2 } } d \vdots f$
		\item From $d,f \in \N, d \vdots f, f \vdots d \Rightarrow d = f$ 
	\end{itemize}
\end{proof}

\begin{property}[]
	If $a,b,c \in \Z, (a,b) = 1 \ \& \ ac \vdots b \Rightarrow c \vdots b$
\end{property}

\begin{proof}[Доказательство]
    Using corollary above (trivial).	
\end{proof}

\begin{property}[]
	Let $a_1, \dots , a_n; b_1, \dots , b_m \in \Z \ \& \ (a_i, b_j) = 1 \Rightarrow (a_1 \dots a_n, b_1 \dots b_m)$
\end{property}

\begin{proof}[Доказательство]
	Using doubled induction.
\end{proof}

\subsection{Prime numbers}

\begin{definition}[]
	\begin{itemize}
		\item Number that has 2 divisors.
		\item Else: factorizable
		\item $P$ -- a set of all primes.
		\item If $p \in P$, then $1 | P, P | P$
		\item $1 \notin P$ 
	 
	\end{itemize}
\end{definition}

\begin{definition}[]
	Let $a \in \N$. Itself divisor of $a$ is any it's divisor, not equal to 1 and $a$. (Intrivial divisort)
\end{definition}

\begin{property}[]
	If $a \in \N$ is factorizable, then $\exists a = bc: b,c \in \N, a > b, c > 1$
\end{property}

\begin{property}[]
	Let $a \in \N, a \neq 1, d$ -- minimal Intrivial divisor of a. Then $d \in P$.
\end{property}

\begin{proof}[Доказательство]
	\begin{itemize}
		\item using definition, $d > 1$
		\item Let $d$ be factorizable. Using corl. 1 $d = bc$, where  $d > b > 1$
		\item From $a \vdots d, d \vdots b \Rightarrow a \vdots b !?$. Then $b < d$ is itself divisor of a   
	\end{itemize}
\end{proof}

\begin{theorem}[]
	There is infinite number of primes.
\end{theorem}

\begin{proof}[Доказательство]
	\begin{itemize}
		\item Let $m = p_1p_2 \dots p_n + 1$, $q$ is minimal itself divisor.
	\end{itemize}
\end{proof}

\begin{property}[]
	Let $a \in \Z, p \in P$. Then $a \vdots p \vee (a,p) = 1$.
\end{property}

\begin{property}[]
	Let $a_1, \dots , a_n \in \Z, p \in P$ such that $a_1 \dots a_n \vdots p \Rightarrow \exists i \in \left\{ 1, \dots , n \right \} $, such that $a_i \vdots p$
\end{property}

\begin{proof}[Доказательство]
	Let not $a_i \vdots p$. Then $(a_i, p) = 1$.

	Using corl. 4,  $(a_1, \dots , a_n, p) = 1$ !?
\end{proof}

\begin{theorem}[OTA]
	$\forall a > 1$ can be factorized into multiplication of primes. This factorization is the only one.
\end{theorem}

\begin{proof}[Доказательство]
	$\exists $. Base $n \in P$ is obvious.
	$ \mapsto $. \begin{itemize}
		\item Let $a \notin P$, and for all $b < a$ theorem is proven.
		\item Then $a = bc$, where $1 < b, c < a \Rightarrow b = p_1 \dots p_n$ and $c = q_1 \dots q_m$
		\item Then $a = p_1 \dots p_n q_1 \dots q_n$ is what we wanted. 
	\end{itemize}

	!. Let $a = p_1 \dots  p_n = q_1 \dots q_m$ -- two factorizations $a$ into prime factorizations, and $a$ is minimal integer, for which factorization is the only one.
	 \begin{itemize}
		\item from $a = p_1 \dots p_n \vdots q_i \Rightarrow p_i \vdots q_1$ for some $i \in \left\{ 1, \dots , n \right \} $. HYO $i = 1$.
		\item From $p_1, q_1 \in P \ \& \ p_1 \vdots q_1 \Rightarrow p_1 = q_1$
		\item Then $a' = \frac{a}{p_1} = p_2 \dots p_n = q_2 \dots q_n$. But factorization $a'$ into multiplication of primes is the only one with the precision of permutations of elements of multiplication. 
		\begin{note}[]
			In particular, $n = m$.
		\end{note}
	\end{itemize}
\end{proof}

\subsection{Canonic factorization}

\begin{definition}[]
	$n = p_1^k_1 p_2^k_2 \dots p_s^k_s$
\end{definition}

\begin{definition}[]
	For $n \in \N$ we denote $d(n)$ as a number of natural divisors $n$.
\end{definition}
