
\section{2023-09-05}

\subsection{Введение}

Пастор -- прочие элементы дискретной математики, в большинстве -- комбинаторика

Карпов -- теория графов

\subsection{Sets and relations}
Reading materials (about sets theory)
\begin{enumerate}
	\item Senpinsky (med)
	\item Vilenkin (easy)
	\item Yekh (hard, ~6th semester)
\end{enumerate}

\href{https://logic.pdmi.ras.ru/~pastor/ITMO/2023-24/}{Slides}


\begin{definition}[Sets]
	множество — это группа объектов (неформально)\end{definition}

\begin{itemize}
	\item objects of a set are its elements
	\item any object can be an element of a set 
	\item $x \in Y$ means <<$x$ is an element of a set $Y$>>
	\item $\emptyset$ is a set that has 0 elements 
\end{itemize}
\begin{itemize}
	\item it's really hard to define set {\bf{formally}}
		\begin{itemize}
		\item formally, sets are also elements of another sets 
		\item other objects have to be interpreted using any sets
	\end{itemize}
\end{itemize}

We are learning {\bf{naive sets theory}}

\subsection{Defining a set}

\begin{itemize}
	\item Перечисление
	\begin{example}[]
		$Y= \left\{ 1,3,7,19,2021 \right\} $
	\end{example}
	\item Defining a set using condition

		$Y = \left\{ x \in X | \ condition \ \right\} $

	\begin{example}[]
		$Y = \left\{ n \in Z | n \ \vdots \ 2 \right\} $
	\end{example}

\end{itemize}
	
\begin{itemize}
	\item If all elements of $Y$ are in $X$, then $Y$ is an subset of $X$. $Y \subset X$
	\begin{note}[]
		$\mathcal{P}(X)$ -- множество всех подмножеств множества $X$
	\end{note}
	\begin{note}
		We can also use $Y = \left\{ x | condition \right\} $, but we call it a class!
	\end{note}
\end{itemize}

\[
	X = \left\{ 1, 2 \right\}\\
		Y = \left\{ 0, \left\{ 1, 2 \right\}, \left\{ 3 \right\}   \right\}\\
		x \in Y, \ but \neg(x \subset Y)\\
\]

\subsection{Проблемы и парадоксы теории множеств}
\begin{itemize}
	\item Can a set be an item of itself?
	\item Russel's paradox (1901): let $Y = \left\{ x | x \notin x \right\}$ 
	\begin{itemize}
		\item is $Y \in Y$?
		\item this question is neither true of false.
	\end{itemize}
	\item in formal sets theory, $x \in x$ is forbidden (regularity axiom)
	\item class $Y$ in Russel's paradox is not a set.
	\item class of every possible set ($U$) is not a set
\end{itemize}

\subsection{Operations between sets}

Let $A, B$ be sets. Then:
	\begin{itemize}
		\item $A \cap B =^{def} \left\{ x \in A | x \in B \right\}$ -- intersection
		\item $A \cup B =^{def} \left\{ x | (x \in A \vee x \in B) \right\}$ -- union
		\item $A \backslash B =^{def} \left\{ x \in A | x \notin B \right\}$ -- difference
		\item $A \triangle B =^{def} \left\{ (A \cup B) \backslash (A \cap B) \right\}	$ -- symmetrical difference
\end{itemize}

$\triangleleft$  Complemention. Let $U$ be an {\it{universum}}, a set that contains every set. then 

$\triangleleft$ $A = ^{def} \left\{ x \in U | x \notin A \right\}$

Упорядоченные пары элементов $x$ $y$ 

\[
	(x,y) = \left\{ (x), (x,y), (y) \right\}
\]

\subsection{Dechart multiplication}

\begin{definition}[]
	$A \times B =^{def} \left\{ (x, y) \ | \ (x \in A \& y \in B) \right\}$ 
\end{definition}


\begin{example}[]
	let $X = \left\{ 1,2,3 \right\}, Y = \left\{ 1,2,3,4,5 \right\}$. Then we got a matrix:
	\[
		\begin{pmatrix} 
		(x_1, y_1) & (x_1, y_2) & ... & (x_1, y_5) \\	
		(x_2, y_1) & (x_2, y_2) & ... & (x_2, y_5) \\	
		(x_3, y_1) & (x_3, y_2) & ... & (x_3, y_5) \\	
	\end{pmatrix}
	\]
\end{example}



Аналогично мы можем определить умножение для $n$ множеств. (можно сделать индукцией)
\[
	A_1 \times A_2 \times \dots \times A_n = \left\{ (x_1,x_2, \dots, x_n | (x_1 \in A_1 \& \dots \& x_n \in A_n)) \right\}
\]

We assume that $A^{1} = A$

If it's not a multiset, then $\left\{ 3, 3 \right\} = \left\{ 3 \right\}$
 
\subsection{Binary relations}

Let $R \subset X \times Y$


If $X = Y$, then  $R$ is a binary relation on $X$.

Lets call a pair $(x,y) : x \in X \& y \in Y$, {\it{appropriate}} 

An appropriate pair $(x,y)$ is written: $xRy$.

\begin{note}[]
	Binary relation is either {\it{true}} or {\it{false}} ($R$ is a set of pairs, for which the statement $x \in X \& y \in Y$ is true)
\end{note}

\begin{note}[]
	Binary relation can be interpreted as an oriented graph: elements of a set is it's vertices, and edges is drawn only if $xRy$.
\end{note}

\subsection{Binary and n-ary relations}
Similarly, $R \subset X_1 \times \dots \times X_n$. If $X_1 = X_2 = \dots = X_n$, then $R$ is an n-ary relation on $X$.

\begin{example}[]
	\begin{enumerate}
		\item Equality ($a = b$) -- binary relation or $\R$ 
		\item divisibility ($a \vdots b$) -- binary relation on $\Z$
		\item let $G = (V, E)$, then:
			 \begin{itemize}
				\item смежность графа --  на $V$.
				\item инцидентность -- между $V, E$.
			\end{itemize}
		\item $A,B,C$ form a line -- 3-ary relation on a plain
	\end{enumerate}
\end{example}

\subsection{Properties of relations}

\begin{definition}[]
	Binary relation is called:
	\begin{itemize}
		\item reflexive, if $xRx$ is true  $\forall x \in X$
		\item irreflexive, if $xRx$ is false  $ \forall x \in X$
		\item symmetrical, if $xRy \Rightarrow yRx$
		\item antisymmetrical, if $xRy \& yRx \Rightarrow x = y$.
		\item transitive, if $xRy, yRz \Rightarrow xRz$ 
	\end{itemize}
\end{definition}

\begin{definition}[]
	Binary relation is relation of identity if it's reflexive, symmetrical and transitive
\end{definition}

\begin{note}[]
	Relation of identity splits the set to {\bf{identity classes}}, so for any 2 elements of 1 class are equal, and 2 elements from different classes aren't
\end{note}

Examples:
\begin{itemize}
	\item $a = b$ 
	\item $a || b$
	\item $a \sim b$
	\item division of polygons by amount of vertices.
\end{itemize}


